\documentclass{article}

\usepackage[utf8]{inputenc}
\usepackage{geometry}
\usepackage{authblk}

\title{UBC Hyperloop -- Vision Statement}

\author[1,*]{Adam Erickson}
\affil[1]{UBC Unmanned Aircraft Systems, Vancouver, BC, V6T 1Z4, Canada}
\affil[*]{E-mail: \texttt{REDACTED}}

\date{August 18, 2019}

\pagenumbering{gobble}

\begin{document}

\maketitle

Reliance on the automobile is likely the single largest drain on public, environmental, and economic health in North America. In the US alone, there are around 270 million vehicles in use (BTS 2018). Personal vehicles are used by 91\% of commuters, who drive a combined total of 11 billion miles per day or 4 trillion miles per year (BTS 2019) -- farther than driving to the Sun and back 43,000 times. In the US, approximately 40,000 people per year, or 110 people per day, are killed in automobile accidents (NHTSA 2019). The fatality rate for automobiles is over 3,000\% higher than trains, 6,600\% higher than buses, and 10,400\% higher than commercial airlines (Savage 2013). The average US driver sits in an automobile for an hour per day, during which time they cannot move, use a bathroom, or be productive, costing 270 million hours per work day in lost productivity.\\

Meanwhile, in the US, 11.1\% of healthcare expenditures are attributable to inadequate physical activity (CDC 2015). Air pollution costs the global economy over \$5 trillion USD per year and causes 10\% of global mortality (World Bank 2016). Poor health costs the US economy \$53 billion USD per year in lost productivity alone (IBI 2018). The average US vehicle costs \$8,500 USD per year to operate (AAA 2018), exceeding the cost of renewable high-speed rail in Europe.\\

In 2018, US energy usage reached a record high of 101.2 quadrillion BTUs (LLNL 2019). About 28\% of this energy is used by the transportation sector, 58\% of which - 16.4 quadrillion BTUs - is used for personal vehicles. A near-total reliance on automobiles carries implications for energy security, global stability, and Earth's climate. This creates land-use patterns of ever-expanding freeway lanes and urban sprawl, yielding large isolated structures and pedestrian-unfriendly communities, imposing secondary costs on energy supply, biodiversity, productivity, and human well-being.\\

Humanity needs a fifth mode of transportation that reduces the use of energy and land, reduces pollution, increases productivity, and improves human well-being. While existing high-speed rail systems are a step in the right direction, further innovation is possible (SpaceX 2013). This will have the added benefit of bolstering economic competitiveness, creating jobs and funding social programs. Join us in the fight to help free humanity from the automobile. We believe that Hyperloop Alpha (SpaceX 2013) is the right starting point for building an environmentally, energetically, and financially sustainable future through human ingenuity.

\end{document}
